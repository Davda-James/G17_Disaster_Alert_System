\documentclass[12pt,a4paper]{article}
\usepackage[utf8]{inputenc}
\usepackage[T1]{fontenc}
\usepackage{geometry}
\usepackage{graphicx}
\usepackage{booktabs}
\usepackage{longtable}
\usepackage{hyperref}
\usepackage{xcolor}
\usepackage{listings}
\usepackage{fancyhdr}
\usepackage{tocloft}

\geometry{margin=1in}

% Colors
\definecolor{codegreen}{rgb}{0,0.6,0}
\definecolor{codegray}{rgb}{0.5,0.5,0.5}
\definecolor{codepurple}{rgb}{0.58,0,0.82}
\definecolor{backcolour}{rgb}{0.95,0.95,0.92}
\definecolor{passgreen}{rgb}{0.2,0.7,0.2}
\definecolor{failred}{rgb}{0.8,0.2,0.2}

% Code listing style
\lstdefinestyle{mystyle}{
    backgroundcolor=\color{backcolour},   
    commentstyle=\color{codegreen},
    keywordstyle=\color{codepurple},
    numberstyle=\tiny\color{codegray},
    stringstyle=\color{codegreen},
    basicstyle=\ttfamily\footnotesize,
    breakatwhitespace=false,         
    breaklines=true,                 
    captionpos=b,                    
    keepspaces=true,                 
    numbers=left,                    
    numbersep=5pt,                  
    showspaces=false,                
    showstringspaces=false,
    showtabs=false,                  
    tabsize=2,
    frame=single
}
\lstset{style=mystyle}

% Header/Footer
\pagestyle{fancy}
\fancyhf{}
\rhead{DisasterWatch Test Procedures}
\lhead{Version 1.0}
\rfoot{Page \thepage}
\lfoot{\today}

\title{
    \vspace{-2cm}
    \textbf{DisasterWatch System}\\
    \Large Test Procedure Document\\
    \vspace{0.5cm}
    \large Software Engineering Assignment 1
}
\author{DAS Development Team}
\date{January 25, 2026}

\begin{document}

\maketitle

\begin{center}
\begin{tabular}{|l|l|}
\hline
\textbf{Document Version} & 1.0 \\
\hline
\textbf{Standard} & IEEE 829 Test Documentation \\
\hline
\textbf{Project} & Disaster Alert System (DAS) \\
\hline
\textbf{Status} & APPROVED \\
\hline
\end{tabular}
\end{center}

\vspace{1cm}

\tableofcontents
\newpage

% ===========================================================================
\section{Overview}
% ===========================================================================

\subsection{System Under Test}

The DisasterWatch application is a mission-critical disaster alert system consisting of:

\begin{table}[h]
\centering
\begin{tabular}{|l|l|l|}
\hline
\textbf{Component} & \textbf{Technology} & \textbf{Port} \\
\hline
Backend API & Flask (Python) & 5000 \\
Frontend & React + Vite & 5173 \\
Database & MongoDB & 27017 \\
SMS Gateway & Twilio API & N/A \\
\hline
\end{tabular}
\caption{System Components}
\end{table}

\subsection{Test Categories Summary}

\begin{table}[h]
\centering
\begin{tabular}{|l|c|c|l|}
\hline
\textbf{Category} & \textbf{Tests} & \textbf{Priority} & \textbf{Description} \\
\hline
Boundary Value & 55 & P1-P2 & Input validation \\
Functional & 12 & P1-P2 & API endpoints \\
Integration & 6 & P1-P2 & End-to-end flows \\
Safety & 10 & P1 & Failure modes \\
Stress & 5 & P1-P2 & Load testing \\
Frontend & 24 & P2-P3 & UI components \\
\hline
\textbf{Total} & \textbf{112} & & \\
\hline
\end{tabular}
\caption{Test Category Summary}
\end{table}

% ===========================================================================
\section{Test Environment Setup}
% ===========================================================================

\subsection{Prerequisites}

Before running tests, ensure you have installed:

\begin{itemize}
    \item Python 3.9 or higher
    \item Node.js 18 or higher
    \item pip (Python package manager)
    \item npm (Node package manager)
\end{itemize}

\subsection{Backend Test Setup}

\textbf{Step 1:} Navigate to project directory
\begin{lstlisting}[language=bash]
cd d:\Sem6_course\SE\Ass1\DAS_Project
\end{lstlisting}

\textbf{Step 2:} Install Backend dependencies
\begin{lstlisting}[language=bash]
pip install -r Backend\requirements.txt
\end{lstlisting}

\textbf{Step 3:} Install Test dependencies
\begin{lstlisting}[language=bash]
pip install -r tests\backend\requirements.txt
\end{lstlisting}

% ...

\subsection{Running All Backend Tests}

\textbf{Command:}
\begin{lstlisting}[language=bash]
cd d:\Sem6_course\SE\Ass1\DAS_Project\tests\backend
pytest -v
\end{lstlisting}

\textbf{Expected Output:}
\begin{lstlisting}
============================= test session starts =============================
platform win32 -- Python 3.11.4, pytest-9.0.2
collected 88 items

boundary/test_limits.py::TestUserInputBoundaries::test_bva001... PASSED
functional/test_alerts.py::TestUserAuthentication::test_ft001... PASSED
...
============================= 88 passed in 5.63s ==============================
\end{lstlisting}

\subsection{Running Specific Test Categories}

\begin{table}[h]
\centering
\begin{tabular}{|l|l|}
\hline
\textbf{Category} & \textbf{Command} \\
\hline
Boundary Value & \texttt{pytest boundary/test\_limits.py -v} \\
Functional & \texttt{pytest functional/test\_alerts.py -v} \\
Integration & \texttt{pytest integration/test\_flow.py -v} \\
Safety & \texttt{pytest safety/test\_failures.py -v -m safety} \\
Stress & \texttt{pytest stress/test\_load.py -v -m stress} \\
\hline
\end{tabular}
\caption{Commands for Specific Test Categories}
\end{table}

\subsection{Running Tests with Coverage}

\begin{lstlisting}[language=bash]
pytest -v --cov=../Backend --cov-report=html
\end{lstlisting}

This generates an HTML coverage report in \texttt{htmlcov/index.html}.

% ===========================================================================
\section{Frontend Tests}
% ===========================================================================

\subsection{Running All Frontend Tests}

\begin{lstlisting}[language=bash]
cd d:\Sem6_course\SE\Ass1\DAS_Project\Frontend
npm run test
\end{lstlisting}

\textbf{Expected Output:}
\begin{lstlisting}
 v src/test/example.test.ts (1 test)
 v src/test/utils.test.ts (15 tests)
 v src/test/AuthContext.test.tsx (8 tests)

 Test Files  3 passed (3)
      Tests  24 passed (24)
   Duration  2.5s
\end{lstlisting}

\subsection{Running Tests in Watch Mode}

\begin{lstlisting}[language=bash]
npm run test:watch
\end{lstlisting}

% ===========================================================================
\section{Notification System Testing}
% ===========================================================================

\subsection{Testing Approach}

The SMS notification system is tested through \textbf{mocking} the Twilio API. No actual SMS messages are sent during testing.

\subsection{Mock Configuration}

\begin{lstlisting}[language=python]
# Tests mock the Twilio Client
with patch('app.Client') as mock_client:
    mock_instance = MagicMock()
    mock_instance.messages.create.return_value = MagicMock(sid='SM123')
    mock_client.return_value = mock_instance
\end{lstlisting}

\subsection{Notification Tests Performed}

\begin{table}[h]
\centering
\begin{tabular}{|l|l|l|}
\hline
\textbf{Test ID} & \textbf{Description} & \textbf{Validates} \\
\hline
FT-009 & SMS triggered for new alert & should\_trigger\_sms() returns True \\
FT-010 & SMS suppressed for duplicate & should\_trigger\_sms() returns False \\
IT-003 & SMS to nearby users & Users within 200km receive SMS \\
IT-006 & Regional distribution & Location-based filtering \\
RBT-001 & Twilio unavailable & Error handling \\
ST-002 & Broadcast throughput & 50+ SMS in <10 seconds \\
\hline
\end{tabular}
\caption{SMS Notification Tests}
\end{table}

\subsection{Notification Logic Diagram}

\begin{verbatim}
1. New Alert Created
   |
   +-- Check: Any SMS sent in area within 12 hours?
   |   +-- YES -> Suppress SMS (sms_sent = False)
   |   +-- NO  -> Continue to Step 2
   |
   +-- Step 2: Find users within 200km radius
       |
       +-- For each user in radius:
       |   +-- Send SMS via Twilio API
       |
       +-- Update alert.sms_sent = True
\end{verbatim}

% ===========================================================================
\section{Test Results Summary}
% ===========================================================================

\subsection{Expected Results}

\begin{table}[h]
\centering
\begin{tabular}{|l|c|c|c|}
\hline
\textbf{Test Suite} & \textbf{Expected} & \textbf{Passed} & \textbf{Status} \\
\hline
Backend - Boundary & 55 & 55 & \textcolor{passgreen}{PASS} \\
Backend - Functional & 12 & 12 & \textcolor{passgreen}{PASS} \\
Backend - Integration & 6 & 6 & \textcolor{passgreen}{PASS} \\
Backend - Safety & 10 & 10 & \textcolor{passgreen}{PASS} \\
Backend - Stress & 5 & 5 & \textcolor{passgreen}{PASS} \\
Frontend & 24 & 24 & \textcolor{passgreen}{PASS} \\
\hline
\textbf{TOTAL} & \textbf{112} & \textbf{112} & \textcolor{passgreen}{\textbf{PASS}} \\
\hline
\end{tabular}
\caption{Final Test Results}
\end{table}

% ===========================================================================
\section{Quick Reference}
% ===========================================================================

\begin{table}[h]
\centering
\begin{tabular}{|l|l|}
\hline
\textbf{Action} & \textbf{Command} \\
\hline
Run all backend tests & \texttt{pytest -v} \\
Run all frontend tests & \texttt{npm run test} \\
Run with coverage & \texttt{pytest --cov=../Backend} \\
Run specific category & \texttt{pytest -m safety} \\
Run single test & \texttt{pytest -k "test\_name"} \\
Watch mode (frontend) & \texttt{npm run test:watch} \\
Generate HTML report & \texttt{pytest --html=report.html} \\
\hline
\end{tabular}
\caption{Quick Reference Commands}
\end{table}

% ===========================================================================
\section{Troubleshooting}
% ===========================================================================

\subsection{Common Issues}

\begin{description}
    \item[Module not found: app] Ensure you're running from the tests directory.
    \item[pytest command not found] Run \texttt{pip install pytest}.
    \item[Connection refused to MongoDB] Tests use mocks - MongoDB is not required.
    \item[Frontend tests failing] Rebuild node\_modules with \texttt{npm install}.
\end{description}

\subsection{Verbose Error Output}

\begin{lstlisting}[language=bash]
pytest -v --tb=long
\end{lstlisting}

% ===========================================================================
\section{Document Approval}
% ===========================================================================

\begin{table}[h]
\centering
\begin{tabular}{|l|l|l|l|}
\hline
\textbf{Role} & \textbf{Name} & \textbf{Signature} & \textbf{Date} \\
\hline
Author & Test Engineer & \_\_\_\_\_\_\_\_\_\_ & 2026-01-25 \\
Reviewer & QA Lead & \_\_\_\_\_\_\_\_\_\_ & \_\_\_\_\_\_\_\_ \\
Approver & Project Manager & \_\_\_\_\_\_\_\_\_\_ & \_\_\_\_\_\_\_\_ \\
\hline
\end{tabular}
\end{table}

\vspace{1cm}
\begin{center}
\textit{--- End of Document ---}
\end{center}

\end{document}
